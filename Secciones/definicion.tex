
\section*{\centering Introducción}

En la actualidad, las tiendas de refacciones para motocicletas enfrentan un desafío 
crítico en la gestión de sus inventarios, lo que impacta directamente en su eficiencia operativa y rentabilidad. 
La falta de un sistema organizado y automatizado para el control de stock genera problemas como desabastecimiento, exceso de mercancía obsoleta y pérdidas económicas significativas.

\subsection*{\centering Problemática Principal}
Uno de los principales obstáculos que enfrentan los propietarios de estos negocios es la ausencia de herramientas eficientes para administrar su inventario. Muchas tiendas aún dependen de métodos manuales, como registros en hojas de cálculo o anotaciones en papel, lo que aumenta el riesgo de errores humanos y dificulta la visibilidad en tiempo real de las existencias. Esto puede llevar a dos escenarios perjudiciales:

\begin{itemize}
    \item \textbf{Falta de productos clave:} Cuando no hay un seguimiento adecuado, se agotan refacciones de alta demanda, generando pérdidas de ventas y clientes insatisfechos.
    \item \textbf{Acumulación de stock innecesario:} La falta de análisis de ventas lleva a sobrecompra de piezas con poca rotación, ocupando espacio y generando costos adicionales.
\end{itemize}

\subsection*{\centering Impacto en los Clientes}
Los clientes suelen enfrentar dificultades para encontrar las refacciones que necesitan debido a:
\begin{itemize}
    \item Desorganización en el almacén.
    \item Falta de personal capacitado.
\end{itemize}

Esto no solo genera frustración, sino que también afecta la reputación del negocio, reduciendo oportunidades de fidelización.

\subsection*{\centering Propuestas de solución}
Para mitigar estos problemas, se recomienda implementar un \textbf{sistema de gestión de inventario especializado} que permita:
\begin{itemize}
    \item Registrar entradas y salidas de productos de forma automatizada.
    \item Analizar tendencias de ventas para optimizar compras.
    \item Agilizar la búsqueda de productos.
\end{itemize}

%%\subsection*{\centering Conclusión}
%%Invertir en tecnología y capacitación mejoraría la eficiencia operativa, aumentaría las ventas y garantizaría la satisfacción del cliente, asegurando la sostenibilidad del negocio a largo plazo.