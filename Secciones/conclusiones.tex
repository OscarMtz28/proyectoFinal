\section*{Conclusiones}

A través del desarrollo de este proyecto se logró comprender de mejor manera ciertos conocimientos previamente adquiridos, como la programación orientada a objetos, la organización en capas y la estructura de un sistema. Durante el proceso se reforzó la comprensión sobre cómo interactúan las distintas capas de un sistema y se comprendió la importancia de mantener el código organizado y estructurado para facilitar su mantenimiento y escalabilidad.

Además, se presentaron algunos retos que permitieron desarrollar habilidades de análisis y resolución de problemas, tales como la configuración de la conexión a la base de datos, la correcta implementación de las entidades y la gestión eficiente de la transferencia de datos entre capas.

A la hora de hacer la conexión a la base de datos fue necesario realizar varias pruebas para asegurar que las conexiones fueran estables y que las consultas se ejecutaran de manera correcta.

A pesar de las dificultades y del tiempo invertido en cada parte de la arquitectura, se logró cumplir con el objetivo. El proyecto no solo funcionó correctamente, sino que también nos dejó una base sólida para futuros desarrollos con estructuras similares.
