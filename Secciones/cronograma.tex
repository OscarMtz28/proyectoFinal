\section*{Cronograma}

\subsection*{Integrantes}
\begin{itemize}
    \item Oscar
    \item Oswaldo
    \item Aaron
\end{itemize}

    \begin{table}[h]
        \begin{center}
            \begin{tabular}{| c | c  | c |}
                \hline
                Fecha & Actividad & Responsable 
                \\
                \hline
                24/03/2025 & Avances 1 & Todos
                \\
                \hline
                31/03/2025 & Avances 2 & Todos 
                \\
                \hline
                07/04/2025 & Codigo & Todos 
                \\
                \hline
                14/04/2025 & Pruebas & Todos 
                \\
                \hline
                21/04/2025 & Entrega & Todos 
                \\
                \hline
            \end{tabular}
        \caption{Cronograma de actividades}
        \end{center}
    \end{table}
\subsection*{Descripción de actividades}

\begin{itemize}
    \item \textbf{Avances 1:} En ese día se mostrarán los primeros avances del proyecto en cuestión con el afán de corroborar si el camino que se vaya tomando sea correcto.
    \item \textbf{Avances 2:} Se presentará el segundo avance. Esta vez incorporando correcciones y mejoras recibidas en la primera entrega.
    \item \textbf{Codigo:} Se presentará una versión con solo algunos detalles por ajustar antes de llegar a la revisión final.
    \item \textbf{Pruebas:} Se hará la entrega del proyecto ya concluido en su totalidad, esto con el afán de detectar errores que pueda tener el mismo, así como también tener en cuenta posibles mejoras que se le puedan realizar.
    \item \textbf{Entrega:} Para este día se hará entrega del proyecto corregido y concluido en su totalidad, listo para su revisión.
\end{itemize}
