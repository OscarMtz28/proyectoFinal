\documentclass{report}

\usepackage[spanish]{babel}
\usepackage{tikz}

\title{Proyecto Final: Tienda de Refacciones de Motocicleta}
\author{Programacion Nacional Socialista}
\date{\today}

\begin{document}

\maketitle


\section*{\centering Introducción}

En la actualidad, las tiendas de refacciones para motocicletas enfrentan un desafío 
crítico en la gestión de sus inventarios, lo que impacta directamente en su eficiencia operativa y rentabilidad. 
La falta de un sistema organizado y automatizado para el control de stock genera problemas como desabastecimiento, exceso de mercancía obsoleta y pérdidas económicas significativas.

\subsection*{\centering Problemática Principal}
Uno de los principales obstáculos que enfrentan los propietarios de estos negocios es la ausencia de herramientas eficientes para administrar su inventario. Muchas tiendas aún dependen de métodos manuales, como registros en hojas de cálculo o anotaciones en papel, lo que aumenta el riesgo de errores humanos y dificulta la visibilidad en tiempo real de las existencias. Esto puede llevar a dos escenarios perjudiciales:

\begin{itemize}
    \item \textbf{Falta de productos clave:} Cuando no hay un seguimiento adecuado, se agotan refacciones de alta demanda, generando pérdidas de ventas y clientes insatisfechos.
    \item \textbf{Acumulación de stock innecesario:} La falta de análisis de ventas lleva a sobrecompra de piezas con poca rotación, ocupando espacio y generando costos adicionales.
\end{itemize}

\subsection*{\centering Impacto en los Clientes}
Los clientes suelen enfrentar dificultades para encontrar las refacciones que necesitan debido a:
\begin{itemize}
    \item Desorganización en el almacén.
    \item Falta de personal capacitado.
\end{itemize}

Esto no solo genera frustración, sino que también afecta la reputación del negocio, reduciendo oportunidades de fidelización.

\subsection*{\centering Propuestas de solución}
Para mitigar estos problemas, se recomienda implementar un \textbf{sistema de gestión de inventario especializado} que permita:
\begin{itemize}
    \item Registrar entradas y salidas de productos de forma automatizada.
    \item Analizar tendencias de ventas para optimizar compras.
    \item Agilizar la búsqueda de productos.
\end{itemize}

%%\subsection*{\centering Conclusión}
%%Invertir en tecnología y capacitación mejoraría la eficiencia operativa, aumentaría las ventas y garantizaría la satisfacción del cliente, asegurando la sostenibilidad del negocio a largo plazo.
\subsection*{\centering Mision}
Desarrollar un \textbf{sistema de gestión de inventario especializado en piezas de motocicleta} que permita a talleres,
distribuidores o usuarios mantener un control preciso y eficiente de su stock, 
facilitando la organización, búsqueda y actualización de repuestos mediante una 
plataforma intuitiva y confiable.

\subsection*{\centering Vision}
Convertirnos en una herramienta esencial para la gestión de inventarios en el sector 
motociclista, ideal para talleres pequeños y medianos, 
con capacidad de adaptarse a futuras necesidades como integración con 
sistemas de ventas, alertas de stock bajo y análisis de demanda.
\newline
\newline
Aunque buscamos un objetivo complejo el proyecto tiene algunas limitaciones:

\begin{itemize}
    \item \textbf{Uso simultaneo:} El proyecto solo permitira el uso de un usario a la vez.
    \item \textbf{Almacenamiento:} La cantidad de piezas a almacenar solo se limitara por la imaginacion del usuario.
    \item \textbf{Busueda:} La interfaz permitira la busqueda de refacciones por nombre o por ID.
    \item \textbf{Sistema Operativo:} El proyecto solo funcionara en sistemas operativos windows.
\end{itemize}

\begin{itemize}
    \item Logra 
\end{itemize}

\section{Solución Propuesta}

Para abordar este problema, proponemos la creación de una tienda de refacciones de motocicleta que ofrezca una amplia gama de productos de alta calidad y un servicio al cliente excepcional. 
La tienda se enfocará en los siguientes aspectos:

\begin{itemize}
    \item \textbf{Inventario extenso:} Mantener un inventario amplio y variado de refacciones para diferentes marcas y modelos de motocicletas.
    \item \textbf{Calidad garantizada:} Ofrecer productos de alta calidad y garantizar la autenticidad de las refacciones.
    \item \textbf{Cantidad de inventario:} Mantener un inventario suficiente para satisfacer la demanda de los clientes y minimizar los tiempos de espera.  
    \item \textbf{Entorno amigable:} Proporcionar un entorno agradable al usuario para facilitar la búsqueda y compra de refacciones. 
    \item \textbf{Eficiencia:} Se busca la eficiencia durante el proceso de compra para que los usuarios puedan encontrar y adquirir las refacciones de manera rápida y sencilla.
\end{itemize}

Con esta solución, esperamos mejorar la experiencia de los propietarios de motocicletas y facilitar el mantenimiento y la reparación de sus vehículos.

\end{document}