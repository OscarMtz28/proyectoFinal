\documentclass{report}

\usepackage[spanish]{babel}
\usepackage{tikz}

\title{Proyecto Final: Tienda de Refacciones de Motocicleta}
\author{Programacion Nacional Socialista}
\date{\today}

\begin{document}

\maketitle

\chapter{Definición del problema}
En la actualidad, las tiendas de refacciones para motocicletas presentan una dificultad grande a la hora del manejo de sus inventarios
llevando estas a su poca eficiencia y tener perdidas en el ingreso. 
\newline Uno de los problemas principales con los que se encuentran los propietarios de este tipo de tiendas es el que no cuentan con una manera 
eficiente de gestionar su inventario, lo que puede llevar a la falta de productos o a la acumulación de refacciones que no se venden.
Además, los clientes pueden tener dificultades para encontrar las refacciones que necesitan, lo que puede resultar en una mala experiencia de compra.
\section*{Mision}

    Alcanzar el posicionamiento como uno de los mejores gestores de inventario
    mejorando siginicativamente la eficiencia de las tiendas de refacciones de motocicleta.

\section*{Vision}

    Ser reconocidos como la mejor opción para la gestión de inventarios de refacciones de motocicleta.

\section*{Alcance} Se que se busca lograr con el proyecto es 
ser uno de los mejores gestores de inventario para las tiendas de refacciones.
\newline
Aunque buscamos un objetivo complejo el proyecto tiene algunas limitaciones:

\begin{itemize}
    \item \textbf{Uso simultaneo:} El proyecto solo permitira el uso de un usario a la vez.
    \item \textbf{Almacenamiento:} El proyecto solo permitira el almacenamiento de 1000 refacciones.
    \item \textbf{Busueda:} La interfaz permitira la busqueda de refacciones por nombre o por ID.
    \item \textbf{Sistema Operativo:} El proyecto solo funcionara en sistemas operativos windows.
\end{itemize}

\begin{itemize}
    \item Logra 
\end{itemize}

\section{Solución Propuesta}

Para abordar este problema, proponemos la creación de una tienda de refacciones de motocicleta que ofrezca una amplia gama de productos de alta calidad y un servicio al cliente excepcional. 
La tienda se enfocará en los siguientes aspectos:

\begin{itemize}
    \item \textbf{Inventario extenso:} Mantener un inventario amplio y variado de refacciones para diferentes marcas y modelos de motocicletas.
    \item \textbf{Calidad garantizada:} Ofrecer productos de alta calidad y garantizar la autenticidad de las refacciones.
    \item \textbf{Cantidad de inventario:} Mantener un inventario suficiente para satisfacer la demanda de los clientes y minimizar los tiempos de espera.  
    \item \textbf{Entorno amigable:} Proporcionar un entorno agradable al usuario para facilitar la búsqueda y compra de refacciones. 
    \item \textbf{Eficiencia:} Se busca la eficiencia durante el proceso de compra para que los usuarios puedan encontrar y adquirir las refacciones de manera rápida y sencilla.
\end{itemize}

Con esta solución, esperamos mejorar la experiencia de los propietarios de motocicletas y facilitar el mantenimiento y la reparación de sus vehículos.

\end{document}